\documentclass[letterpaper,12pt]{article}
\usepackage{array}
\usepackage{threeparttable}
\usepackage{geometry}
\geometry{letterpaper,tmargin=1in,bmargin=1in,lmargin=1.25in,rmargin=1.25in}
\usepackage{fancyhdr,lastpage}
\pagestyle{fancy}
\lhead{}
\chead{}
\rhead{}
\lfoot{}
\cfoot{}
\rfoot{\footnotesize\textsl{Page \thepage\ of \pageref{LastPage}}}
\renewcommand\headrulewidth{0pt}
\renewcommand\footrulewidth{0pt}
\usepackage[format=hang,font=normalsize,labelfont=bf]{caption}
\usepackage{listings}
\lstset{frame=single,
  language=Python,
  showstringspaces=false,
  columns=flexible,
  basicstyle={\small\ttfamily},
  numbers=none,
  breaklines=true,
  breakatwhitespace=true
  tabsize=3
}
\usepackage{amsmath}
\usepackage{amssymb}
\usepackage{amsthm}
\usepackage{harvard}
\usepackage{setspace}
\usepackage{float,color}
\usepackage[pdftex]{graphicx}
\usepackage{hyperref}
\hypersetup{colorlinks,linkcolor=red,urlcolor=blue}
\theoremstyle{definition}
\newtheorem{theorem}{Theorem}
\newtheorem{acknowledgement}[theorem]{Acknowledgement}
\newtheorem{algorithm}[theorem]{Algorithm}
\newtheorem{axiom}[theorem]{Axiom}
\newtheorem{case}[theorem]{Case}
\newtheorem{claim}[theorem]{Claim}
\newtheorem{conclusion}[theorem]{Conclusion}
\newtheorem{condition}[theorem]{Condition}
\newtheorem{conjecture}[theorem]{Conjecture}
\newtheorem{corollary}[theorem]{Corollary}
\newtheorem{criterion}[theorem]{Criterion}
\newtheorem{definition}[theorem]{Definition}
\newtheorem{derivation}{Derivation} % Number derivations on their own
\newtheorem{example}[theorem]{Example}
\newtheorem{exercise}[theorem]{Exercise}
\newtheorem{lemma}[theorem]{Lemma}
\newtheorem{notation}[theorem]{Notation}
\newtheorem{problem}[theorem]{Problem}
\newtheorem{proposition}{Proposition} % Number propositions on their own
\newtheorem{remark}[theorem]{Remark}
\newtheorem{solution}[theorem]{Solution}
\newtheorem{summary}[theorem]{Summary}
%\numberwithin{equation}{section}
\bibliographystyle{aer}
\newcommand\ve{\varepsilon}
\newcommand\boldline{\arrayrulewidth{1pt}\hline}


\begin{document}

\begin{flushleft}
  \textbf{\large{Homework 1: Dynamic Programming}} \\
  OSE Lab 2019, Prof.\ dr.\ Felix Kubler \\
  Wouter van der Wielen
\end{flushleft}

\vspace{5mm}

\noindent\textbf{Problem 1}
\textbf{(Asset market equilibrium)}
Each agent $h$ will maximize their utility
\begin{equation} \label{problem}
	\max_{\{c_0, ..., c_4\}} U^{h}(c) = \max_{c} \left[ \frac{c_0^{1-\gamma}}{1-\gamma} + \frac{1}{4}\sum_{s=1}^{4} \frac{c_s^{1-\gamma}}{1-\gamma} \right]
\end{equation}

\noindent over two periods subject to the budget constraint
\begin{equation}
	c_0^h = e_0^h - q_1\theta_1^h - q_2\theta_2^h \hspace{20pt} \text{at } t = 0
\end{equation}
\noindent and
\begin{equation}
	c_s^h = e_s^h + \theta_1^h A_s^1 + \theta_2^h A_s^2 \hspace{20pt} \text{otherwise}
\end{equation}
\noindent where $q_i$ the initial price of asset $i$, $e_s^h$ is agent $h$ his endowment (note, $e_0=1$ for both agents), $\theta_i^h$ his holdings of asset $i$ and $A_s^i$ the pay out from asset $i$ in state $s$. Moreover, the following market clearing conditions have to hold:
\begin{align}
	\theta_1^1 + \theta_1^2 = 0 \\
	\theta_2^1 + \theta_2^2 = 0
\end{align}

We can plug the budget constraints and market clearing conditions 
directly into \eqref{problem}. Consequently, the First Order Conditions (FOCs) are as follows:

\begin{equation}
	\frac{\partial U^1}{\partial \theta_1^1} = -q_1(e_0^1 - q_1\theta_1^1 - q_2\theta_2^1)^{-\gamma} + \frac{1}{4}\sum_{s=1}^{4} A_s^1(e_s^1 + \theta_1^1 A_s^1 + \theta_2^1 A_s^2)^{-\gamma} = 0
\end{equation}
\begin{equation}
	\frac{\partial U^1}{\partial \theta_2^1} = -q_2(e_0^1 - q_1\theta_1^1 - q_2\theta_2^1)^{-\gamma} + \frac{1}{4}\sum_{s=1}^{4} A_s^2(e_s^1 + \theta_1^1 A_s^1 + \theta_2^1 A_s^2)^{-\gamma} = 0
\end{equation}
\begin{align}
	\frac{\partial U^2}{\partial \theta_1^2} &= -q_1(e_0^2 - q_1\theta_1^2 - q_2\theta_2^2)^{-\gamma} + \frac{1}{4}\sum_{s=1}^{4} A_s^1(e_s^2 + \theta_1^2 A_s^1 + \theta_2^2 A_s^2)^{-\gamma} \nonumber \\
	&= -q_1(e_0^2 + q_1\theta_1^1 + q_2\theta_2^1)^{-\gamma} + \frac{1}{4}\sum_{s=1}^{4} A_s^1(e_s^2 - \theta_1^1 A_s^1 - \theta_2^1 A_s^2)^{-\gamma} = 0
\end{align}
\begin{align}
	\frac{\partial U^2}{\partial \theta_2^2} &= -q_2(e_0^2 - q_1\theta_1^2 - q_2\theta_2^2)^{-\gamma} + \frac{1}{4}\sum_{s=1}^{4} A_s^2(e_s^2 + \theta_1^2 A_s^1 + \theta_2^2 A_s^2)^{-\gamma} \nonumber \\
	&= -q_2(e_0^2 + q_1\theta_1^1 + q_2\theta_2^1)^{-\gamma} + \frac{1}{4}\sum_{s=1}^{4} A_s^2(e_s^2 - \theta_1^1 A_s^1 - \theta_2^1 A_s^2)^{-\gamma} = 0
\end{align}

\noindent Hence, resulting in a system of 4 equations in 4 unknowns ($q_1$, $q_2$, $\theta_1^1$, $\theta_2^1$)

\end{document}

